% !TeX root = ../thuthesis-example.tex

% 中英文摘要和关键字

\begin{abstract}
  论文的摘要是对论文研究内容和成果的高度概括。
  摘要应对论文所研究的问题及其研究目的进行描述,对研究方法和过程进行简单介绍,对研究成果和所得结论进行概括。
  摘要应具有独立性和自明性,其内容应包含与论文全文同等量的主要信息。
  使读者即使不阅读全文,通过摘要就能了解论文的总体内容和主要成果。

  论文摘要的书写应力求精确、简明。
  切忌写成对论文书写内容进行提要的形式,尤其要避免“第 1 章……;第 2 章……;……”这种或类似的陈述方式。

  关键词是为了文献标引工作、用以表示全文主要内容信息的单词或术语。
  关键词不超过 5 个,每个关键词中间用分号分隔。

  % 关键词用“英文逗号”分隔,输出时会自动处理为正确的分隔符
  \thusetup{
    keywords = {关键词 1, 关键词 2, 关键词 3, 关键词 4, 关键词 5},
  }
\end{abstract}

\begin{abstract*}

    Abstract: This paper presents the T-RexNet approach to detect small moving objects in videos by
using a deep neural network. T-RexNet combines the advantages of Single-Shot-Detectors with
a specific feature-extraction network, thus overcoming the known shortcomings of Single-ShotDetectors in detecting small objects. The deep convolutional neural network includes two parallel
paths: the first path processes both the original picture, in gray-scale format, and differences between
consecutive frames; in the second path, differences between a set of three consecutive frames is only
handled. As compared with generic object detectors, the method limits the depth of the convolutional
network to make it less sensible to high-level features and easier to train on small objects. The simple,
Hardware-efficient architecture attains its highest accuracy in the presence of videos with static
framing. Deploying our architecture on the NVIDIA Jetson Nano edge-device shows its suitability to
embedded systems. To prove the effectiveness and general applicability of the approach, real-world
tests assessed the method performances in different scenarios, namely, aerial surveillance with the
WPAFB 2009 dataset, civilian surveillance using the Chinese University of Hong Kong (CUHK)
Square dataset, and fast tennis-ball tracking, involving a custom dataset. Experimental results prove
that T-RexNet is a valid, general solution to detect small moving objects, which outperforms in
this task generic existing object-detection approaches. The method also compares favourably with
application-specific approaches in terms of the accuracy vs. speed trade-off.


  % Use comma as separator when inputting
  \thusetup{
    keywords* = {keyword 1, keyword 2, keyword 3, keyword 4, keyword 5},
  }
\end{abstract*}
